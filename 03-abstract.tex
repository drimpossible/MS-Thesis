\noindent Deep learning models have evolved remarkably, and are pushing the state-of-the-art in various problems across domains. At the same time, the complexity and the amount of resources these DNNs consume has greatly increased. Today's DNNs are computationally intensive to train and run, especially Convolutional Neural Networks (CNNs) used for vision applications. They also occupy a large amount of memory and consume a large amount of power during training. This poses a major roadblock to the deployment of such networks, especially in real-time applications or on resource-limited devices. Two methods have shown promise in compressing CNNs: (i) Binarization and (ii) Pruning. We explore these two methods in this thesis.\\

\noindent The first method to achieve improvements in computational/spatial efficiency is to binarize (1-bit quantize) the weights and activations in a network. However, naive binarization results in accuracy drops for most tasks. In this work, we present a Distribution-Aware approach to Binarizing Networks (DABN) that allows us to retain the advantages of a binarized network, while improving accuracy over binary networks. We also develop efficient implementations of DABN across different architectures. We present a theoretical analysis of DABN to show the effective representational power of the resulting layers, and explore the forms of data they model best. Experiments on popular sketch datasets show that DABN offers better accuracies than naive binarization. \\

\noindent We further investigate the question of where to binarize inputs at layer-level granularity and show that selectively binarizing the inputs to specific layers in the network could lead to significant improvements in accuracy while preserving most of the advantages of binarization. We analyze the binarization tradeoff using a metric that jointly models the input binarization error and computational cost. We introduce an efficient algorithm to select layers whose inputs are to be binarized. We discuss practical guidelines based on insights obtained from applying the algorithm to a variety of models. Experiments on the Imagenet dataset using AlexNet and ResNet-18 models show 3-4\% improvement in accuracy over fully binarized networks with minimal impact on compression. The improvements are even more substantial on sketch datasets like TU-Berlin, where we match state-of-the-art accuracy, getting more than 8\% increase in accuracies over binary networks. We further show that our approach can be applied in tandem with other forms of compression that deal with individual layers or overall model compression (e.g., SqueezeNets). In contrast to previous binarization approaches, we are able to binarize the weights in the last layers of a network, which enables us to compress a large fraction of additional parameters. \\

\noindent The second method explored is pruning. We investigate pruning neural networks from a graph-theoretic perspective. Efficient CNN designs like ResNets and DenseNet were proposed to improve accuracy vs efficiency trade-offs. They essentially increased the connectivity, allowing efficient information flow across layers. Inspired by these techniques, we propose to model connections between filters of a CNN using graphs which are simultaneously sparse and well-connected. Sparsity results in efficiency while well-connectedness can preserve the expressive power of the CNNs. We use a well-studied class of graphs from theoretical computer science that satisfies these properties known as Expander graphs. Expander graphs are used to model connections between filters in CNNs to design networks called X-Nets. We present two guarantees on the connectivity of X-Nets: (i) Each node of a layer influences every node in a layer $O(logn)$ steps away, where $n$ is the number of layers between the two layers (ii) The number of paths between two sets of nodes is proportional to the product of their sizes. We also propose efficient training and inference algorithms, making it possible to train deeper and wider X-Nets effectively. Expander based models give a 4\% improvement in accuracy on MobileNet over grouped convolutions, a popular technique which has the same sparsity but worse connectivity. X-Nets give better performance trade-offs than the original ResNet and DenseNet-BC architectures. We achieve model sizes comparable to state-of-the-art pruning techniques using our simple architecture design, without any pruning. 