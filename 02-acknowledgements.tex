\noindent I was very lucky to have Riddhiman Dasgupta as my mentor. He was the one who helped me take my first steps as a researcher back in 2015. Along with Koustav Ghosal, they were the instrumental duo in guiding me through ``research 101". I would like to thank my friends, we called ourselves the ``The Roaches": Sri Aurobindo Munagala, Rohit Gajawada and Vishal Batchu who made this journey memorable, the times we attacked challenging problems and won some battles. I thank all the folks in CVIT, for our deep dive into topics and coffee-nights. Exploring silly ideas since 2015 somehow ended up in this thesis.\\

\noindent Unlike the writing of this thesis, which is usually an exercise in sustained suffering, the journey itself was one of my most memorable. I thank my best friend Shreedhar Manek, without whom I cannot imagine my college life. I've spent some of the best times with him: traveled mountains, explored jungles, become vegan. Along with my friends Shantanu, Dhruv, Aalekh, Anshul and other wingmates- the sole undivided wing. I would like to thank Auro, Anurag, Tushant, Arjun, Saujas, Alok and Shyam who awoke in me curiosity for philosophy, awareness of social and economic issues, awe of the great historical traditions, good writing and poetry among other things.\\

\noindent I love research. However, I largely attribute success in my research to luck. I thank the metaphorical God for the opportunity of making me the coin flip which won the lottery at every single step, so many times. This was augmented by some important causal factors. The largest of them are: Dr. Anoop Namboodiri provided me with the guidance, support, and tremendous amounts of freedom to explore. Dr. Maneesh Singh taught me how to think about a problem, a principled way of doing research: asking the right questions, identifying real issues, dissuade random exploration and post-facto reasoning. Dr. Manish Shrivastava taught me a respect for language (domain) while applying ML algorithms, and the resulting elegance when combined correctly. Actually, let's say they tried their best. I hope I learned. Yes, looking back they all displayed one quality: incredible patience in tolerating my youthful pride and ignorance. These four years of research, more than anything, have been a humbling experience. \\

\noindent I now see a glimpse of the sheer depth of vision, linguistics, statistics and machine learning literature. I realize how long a path lies ahead and have the determination to walk it with the goal of helping to alleviate some of the most pressing problems people face in the world. 
